\documentclass{article}

\usepackage{float}
\usepackage{hyperref}
\usepackage{url}
\usepackage{amsmath,amssymb}

\include{vc}

\pagestyle{empty}

\baselineskip 18pt
\textwidth 6.25in
\textheight 8.5in
\oddsidemargin 0.1in
\evensidemargin 0.1in
\marginparwidth 0in
\marginparsep 0in
\topmargin -.5in
\topskip -.5in
\parindent 0in
\parskip 1pt

\begin{document}

\begin{center}
  \LARGE{\scshape{AST1420: Galactic structure and dynamics}}\\[5pt]
  \Large{\scshape{Fall 2017}}\\[5pt]
  \large{(last updated: \today; rev. \githash)}\\[25pt]
\end{center}

\section*{Course description}

This course will consist of the following.

\section*{Logistics}

\begin{itemize}

  \item {\bf Meeting time:} TBD

  \item {\bf Instructor:} Jo Bovy, AB 229.

  \item {\bf Email:} \href{mailto:jo.bovy@utoronto.ca}{jo.bovy@utoronto.ca}

  \item {\bf Office hours:} Stop by my office or by appointment.

\end{itemize}

\section*{Learning objectives}

\section*{Reading}

We will mostly be using

\begin{itemize}

  \item Binney \& Tremaine, \emph{Galactic Dynamics, 2nd Edition},
    2008, Princeton University Press. Errata can be found
    \href{https://www-thphys.physics.ox.ac.uk/people/JamesBinney/web/index\_files/BT2errors.pdf}{here}.

\end{itemize}

We will also be referring to

\begin{itemize}

  \item Binney \& Merrifield, \emph{Galactic Astronomy},
    1998, Princeton University Press. Errata can be found
    \href{http://www-thphys.physics.ox.ac.uk/people/JamesBinney/bmerrors.pdf}{here}.

\end{itemize}

Other useful textbooks are

\begin{itemize}

  \item Schneider, \emph{“Extragalactic Astronomy and Cosmology}, 2015, Springer. You can access this book online on the UofT wireless network by going to:
{\footnotesize \begin{center} \url{http://link.springer.com/book/10.1007%2F978-3-642-54083-7}~.\end{center}}

\end{itemize}

\section*{Grading scheme}

\begin{itemize}

  \item {\bf Assignments:} 40\,\%

  \item {\bf Presentations:} 30\,\%

  \item {\bf Take-home final + oral exam:} 30\,\%

\end{itemize}

\section*{Academic integrity}

From Appendix D of the Academic Integrity Handbook:
\begin{quote}
  Academic integrity is one of the cornerstones of the University of
  Toronto. It is critically important both to maintain our community
  which honours the values of honesty, trust, respect, fairness, and
  responsibility and to protect you, the students within this
  community, and the value of the degree towards which you are all
  working so diligently.  

  According to Section B of the University of
  Toronto's Code of Behaviour on Academic Matter
  (\url{http://www.governingcouncil.utoronto.ca/policies/behaveac.htm})
  which all students are expected to read and by which they are
  expected to abide, it is an offence for students to:
  \begin{itemize}
    \item Use someone else's ideas or words in their own work without
      acknowledging explicitly that those ideas/words are not their
      own with a citation and quotation marks, i.e. to commit
      plagiarism.
  \item Include false, misleading, or concocted citations in their
    work.
  \item Obtain unauthorized assistance on any assignment. 
  \item Provide unauthorized assistance to another students. This
    includes showing another student your own work.
  \item Submit their own work for credit in more than one course
      without the permission of the instructors.
  \end{itemize}

  There are other offenses covered under the Code, but these are the
  most common. You are instructed to respect these rules and the
  values that they protect.
\end{quote}

\newpage

\section*{Schedule}

\begin{itemize}

  \item {\bf Week 1:} Introduction to galactic structure; coordinate
    systems; galaxies as collisionless sytems.

  \item {\bf Week 2:} Properties of spherical mass distribution;
    orbits in spherical potentials.

  \item {\bf Week 3:} Equilibrium configurations of spherical systems;
    virial theorem; spherical Jeans equations; spherical distribution
    functions; applications to dwarf spheroidal galaxies, globular
    clusters, galaxy clusters.

  \item {\bf Week 4:} The Local Group; the Hubble sequence;
    observational properties of elliptical and disk galaxies.

  \item {\bf Week 5:} Properties of flattened mass distributions;
    orbits in axisymmetric potentials.

  \item {\bf Week 6:} Rotation curves; gas kinematics in the Milky
    Way; the Milky Way's rotation curve; Oort constants.

  \item {\bf Week 7:} Equilibrium configurations of disks;
    axisymmetric Jeans equations; disk distribution functions; the
    dynamics of the Solar neighborhood.

  \item {\bf Week 8:} Triaxial mass distributions; Schwarzschild
    modeling; the internal structure of elliptical galaxies.

  \item {\bf Week 9:} $N$-body modeling; chemical evolution of
    galaxies; age--abundance relations in the solar neighborhood;
    stellar population synthesis.

  \item {\bf Week 10:} Dynamical instabilities; bars; spiral arms;
 mergers and dynamical friction; violent relaxation; phase-mixing.

  \item {\bf Week 11:} Tides; high-speed encounters; dispersal of open
    clusters; adiabatic contraction. Scattering of stars in disk.

  \item {\bf Week 12:} Presentations ({\bf IN AN EARLIER WEEK}).

\end{itemize}


\end{document}

